% must be between 150 and 250, so aiming for 200!
\begin{abstract}
The use of head-mounted virtual reality (VR) to induce presence in a computer simulated world has proven to significantly increase the ecological validity of this medium. In VR, illusions of various kinds (place illusion, plausibility illusion, etc.) occur at the same time for the user to feel present. Most prominently, the embodiment illusion has proven to elicit 'realistic' behavioral as well as physiological responses, when a strong emotional stimulus such as virtual hurting of an embodied rubber hand is provided. Yet, to be able to employ VR and claim ecological validity for less emotional stimuli, the level of presence must be accounted for. We show that the level of presence impacts free spatial exploration behavior in a large scale VR navigation paradigm. Here, we investigate the impact of an established presence metric on ongoing motor behavior demonstrating an analysis framework with a high spatial resolution. We observed participants with higher presence to stay closer to the walls while exploring invisible mazes. Ultimately, we link presence to individual differences in video game experience, sex, and spatial perspective taking abilities, confirming that user characteristics are a defining part of the presence construct.
\end{abstract}
% in the end use hemingway app (http://www.hemingwayapp.com) to increase readability
% Word count is 194: