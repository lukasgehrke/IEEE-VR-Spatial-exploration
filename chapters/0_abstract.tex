The use of head-mounted virtual reality to induce illusions of various kinds has proven to significantly increase the ecological validity of research in the cognitive sciences. Most prominently, the embodiment illusion has provided first insights into the 'homuncular' flexibility of human cognition. Strong emotional stimuli, such as virtual hurting of an embodied rubber hand, have proven to elicit 'realistic' responses, both in terms of behavior and physiological responses. Yet, to be able to employ virtual reality for less emotional stimuli, the subjective strength of the ongoing illusion has to be accounted for. Here, we shed further light on on individual differences in presence experienced by users in the same virtual world by demonstrating that presence can be predicted with high accuracy by video game experience, sex, and spatial perspective taking abilities. Furthermore, we show that the level of presence impacts free spatial exploration behavior in a large scale VR during a spatial navigation paradigm. Using a methodological framework developed for the cognitive neurosciences provides a powerful analysis framework to investigating ongoing behavior in cognitive paradigms in greater detail.