% \begin{abstract}
\abstract{Genuine experiences where users feel a deep level of connection are the key quality of room-scale virtual reality (VR). The freedom to move promises natural sensory experiences stimulating a feeling of presence. However, users differ in their eagerness to move, some prefer movement by teleportation while others would keep walking forever. Such individual differences challenge the inclusive design necessary for bestseller applications. In this methodological research contribution, we propose to study user behavior and experience using parametric maps based on general linear models (GLM) to overcome limitations of traditional data aggregation techniques. In the investigated study, participants explored invisible mazes touching hidden walls for brief moments of visual guidance. We demonstrate that experienced presence correlated with where participants spent time exploring the VR. We found an increase in presence coinciding with participants being less likely to collide with invisible walls and spending more time in segments critical for navigational success.}
% \end{abstract}

%old
%The use of head-mounted virtual reality (VR) to induce presence in a computer simulated world has proven to significantly increase the ecological validity of this medium. In VR, illusions of various kinds (place illusion, plausibility illusion, etc.) occur at the same time for the user to feel present. Most prominently, the embodiment illusion has proven to elicit 'realistic' behavioral as well as physiological responses, when a strong emotional stimulus such as virtual hurting of an embodied rubber hand is provided. Yet, to be able to employ VR and claim ecological validity for less emotional stimuli, the level of presence must be accounted for.
%Designing remote interactions that induce truly connected social experiences is one of the great promises of room-scale VR. 
%We show that the level of presence impacts free spatial exploration behavior in a large scale VR navigation paradigm. Here, we investigate the impact of ongoing motor behavior with an established presence metric demonstrating an analysis framework with high resolution of user behavior in VR space. We observed participants with higher presence to exhibit higher spatial awareness by spending more time at navigationally relevant segments. Ultimately, the analysis approach links presence to individual differences in video game experience, sex, and spatial perspective taking abilities, confirming that user characteristics are a defining part of the presence construct.
%Video game experience, biological sex as well as spatial perspective taking skills proved to be significant predictors of a rich presence experience and were able to predict presence to within 3/4 of a point accuracy on a standard questionnaire scale. 
