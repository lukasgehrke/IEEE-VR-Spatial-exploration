\subsection{Presence and spatial exploration behavior}

One of the key VR illusions contributing to the subjective construct of presence is the place illusion which can be defined as the perception of oneself being present in a virtual place where one can act, react, and impact the surroundings \cite{Slater2009}. Therefore, self-location, sense of agency, and the spatial awareness of the surroundings are strongly impacted by the place illusion and modify behavior \cite{Kilteni2012}. When one perceives oneself as in control of their own actions and observe action consequences in their virtual surroundings, spatial exploration behavior becomes a part of a learning process to adapt motor behavior to the surroundings as opposed to a random chain of actions executed by the user to explore, for instance, only the VR technology itself \cite{Tan2011}.

% this is the important motivation section 

% Gonzales-Franco 'Model of Illusions'