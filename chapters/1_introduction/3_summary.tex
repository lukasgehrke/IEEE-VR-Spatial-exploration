\subsection{Summary and Hypotheses}
In the present work, we address whether the level of experienced presence affects 'realistic' ongoing motor behavior, i.e. spatial exploration behavior in a large VR mimicking a real-world exploration situation such as finding your way in complete darkness. Here, we investigated the impact of experienced presence on spatial exploration behavior in the invisible maze task~\cite{Gehrke2018}. To address our motivation, we conducted the following two-step analysis. In a first step, we built a linear model to predict experienced presence given per participant descriptors. We were primarily interested to see how accurate presence can be predicted from broad knowledge of participant movement behavior. Additionally we considered a multitude of participant descriptors to derive a useful model. We reduced the model to a minimum of useful predictors so other researchers may easily reproduce our findings. In other words, we addressed how accurate subjectively reported presence may be predicted given a number of per participant descriptors.
In a second step, we increased the resolution of our analyses to specifically investigate ongoing motor behavior. Here, we analysed in detail whether participants movement behavior differed as a function of experienced presence and where it did so. To this end, we conducted mass-univariate pixel-by-pixel modeling of experienced presence on duration spent in a certain location and the number of wall touches elicited there, the simple where and what of participants actions.

%Here, we propose a highly accurate subjective presence metric based on individual differences in video game experience, sex, and spatial perspective taking abilities. Furthermore, we show that the level of presence impacts free spatial exploration behavior in a large scale VR during a spatial navigation paradigm. Finally, using a methodological framework developed for the cognitive neuroscience, we showcase a powerful analysis framework to investigate ongoing behavior in cognitive paradigms in greater detail.

%Finally, using a methodological framework developed for the cognitive neuroscience, we showcase a powerful analysis framework to investigate ongoing behavior in cognitive paradigms in greater detail.