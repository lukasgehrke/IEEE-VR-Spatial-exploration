Using virtual reality (VR) for neuroscientific research has gained serious traction over the last years with the advent of affordable VR systems and lightweight sensors. Immersive paradigms make use of illusions to subject participants to controlled, yet stimulus rich experiences \cite{Bohil2011}. Many researchers argue for an increase in ecological validity by increasing the amount of congruent sensory information. Yet, only a few studies have investigated the impact of the effectiveness of VR illusions on rich behavioral, psychometric and bio-physiological parameters \cite{Slater2009, Parsons2015, Kobayashi2015, Gonzalez-Franco2017}. With all modern virtual reality headsets providing precise motion capture, synchronized data collection is easily accomplished and allows to address rich, ecological valid, behavior taking place under VR illusions \cite{MakeigS.GramannK.JungT.P.SejnowskiT.J.Poizner2009, Gramann2014}. To date, researchers reported 'realistic' behavior as an approximation of ecologically valid behavior under VR illusions /. However, whether the effectiveness of VR illusion affects ongoing motor behavior under detailed investigation is rarely addressed and poses significant challenges to the ecological validity of VR as an investigative tool in cognitive science and neuroscience.

% we do not address whether participants “treat what they perceive as real” and feel present in the virtual world (Slater, 2009).
\colorbox{red}{Actually the argumentation is like this:}
\begin{itemize}
    \item first we wanted to see if we can predict presence using participant average responses and psychological and psychometric descriptors, i need a better word for what biological sex, video game experience and ptsot etc are
    \item we reduced our model to the best explaining factors so other researchers may best benefit from our generalizable findings
    \item we realized that using participant averages for ongoing motor behavior is bad because may we may loose a lot of relevant information
    \item therefore we introduce the novel analyses with a fine spatial resolution to investigate the impact the strength of the place illusion has on the ongoing motor behavior
\end{itemize}