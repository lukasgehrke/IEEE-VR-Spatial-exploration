% What problem are you solving, or opportunity are you seizing? 
\section{1 intro}
Especially in times of a global pandemic, increasing the (psychological) depth of people's remote connections is desirable across the board. Room-scale VR significantly increases the users immersion by, first and foremost, allowing free movements, simulating the sensory \textit{natural} experience in the real world. Further, synchronized motion capture allows avatar movements to be rendered equal to the own bodies movements, inducing the illusion of body ownership transfer \cite{Kilteni2012}. Taken together, free movement, \textit{realistic} avatar rendering and contextual components of the VR experience coincide at the same time for the user to feel present \cite{Gonzalez-Franco2017}. Therefore, designing immersive experiences for room-scale VR aims at facilitating the emergence of presence experience, dissolving the feeling of connectedness to the real body, ultimately providing the foundation for genuinely connected (social) experiences. 

However, in order to scale immersive VR technology to a broader public with use cases ranging from remote office work to entertainment, inclusive design principles are of key importance to successfully design presence experience across the user base. Individual differences, for example the proclivity to move throughout virtual worlds, significantly challenge designing for presence experience thereby challenging acceptance of VR technology in general. Here, designers and developers would benefit from a rich understanding of user behavior, being able to directly query the impact of certain characteristics of their, unintended, neglected user base. Specifically in room-scale VR applications, leveraging inherent motion capture provides the opportunity for data-driven understanding of how individual proclivities explain user experience with spatial specificity. However, typical data reports usually consider aggregate data missing the opportunity to spatially resolve effects of interest.

%%%%% writing ressources
%As a concurrent effect, we present an easily adaptable and scalable GLM analyses framework to investigate movement behavior with a high spatial resolution.
%Here, we first confirm individual differences in movement profiles of experienced video gamers moving faster and more efficient when exploring a large-scale VR.
