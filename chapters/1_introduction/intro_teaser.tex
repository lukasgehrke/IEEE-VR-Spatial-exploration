% What problem are you solving, or opportunity are you seizing? 
\section{1 intro}
Especially in times of a global pandemic, increasing the (psychological) depth of people's remote connections is desirable across the board. Room-scale VR significantly increases the users immersion by, first and foremost, allowing free movements, simulating the \textit{natural} experience in the real world. Designing immersive experiences for room-scale VR aims at inducing presence experience. Dissolving the feeling of connectedness to the real body, transferring the self to avatar renderings, ultimately provides the foundation for genuinely connected (social) experiences in room-scale VR.

In order to scale immersive VR technology to a broader public with use cases ranging from remote office work to entertainment, inclusive design principles are of key importance to induce presence experiences across the user base. However, individual differences, for example the proclivity to move, significantly challenge designing presence experience as well as VR technology acceptance in general. Fortunately, VR technology through its inherent motion capture provides the opportunity to study user behavior with a rich spatial resolution, subsequently advising developers and designers with a data-driven understanding of how individual proclivities underlie user experience. However, data analyses procedures usually consider aggregate data missing the opportunity to spatially resolve effects of interest.
% citations


%%%%% writing ressources
%As a concurrent effect, we present an easily adaptable and scalable GLM analyses framework to investigate movement behavior with a high spatial resolution.
%Here, we first confirm individual differences in movement profiles of experienced video gamers moving faster and more efficient when exploring a large-scale VR.
