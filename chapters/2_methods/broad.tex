\subsection{Statistical Analyses}
\subsubsection{Predicting Presence using Participant Descriptives} First, we were motivated to test the accuracy of predicted presence scores using general participant descriptives. We computed an ordinary least squares regression entering IPQ presence scores as the dependent variable using R \cite{RFoundationforStatisticalComputing.2018}. For the predictors, we first computed the participant average across mazes and runs, see \cite{Gehrke2018} and then entered: movement velocity, exploration duration, number of wall touches, sketch map accuracy, video game experience, sex, perspective taking and orientation ability and lastly sense of direction into the regression model. For an explanation of each predictor, please consult \cite{Gehrke2018}. To reduce over-fitting and increase the possible insight of our results for other researchers through generalization, a step-wise model selection procedure based on Akaike's information criterion (AIC) was computed using 'stepAIC' of package 'MASS' \cite{Akaike1998a, Venables2002}.
Ultimately, the reduced model of three predictors was assessed in terms of its predictive accuracy. Therefore, a cross-validation with 5 folds was computed to obtain a robust mean absolute error \cite{Mosteller1968, Furnkranz2011}. With 29 participants comprising the analyzed data, each training fold consisted of either 23 or 24 participants with either 5 or 6 participants in the evaluated test fold. The R code and data used are available online\footnote{https://github.com/lukasgehrke/2019-IEEE-VR-Spatial-exploration-behavior-in-large-scale-VR-predicts-subjective-spatial-presence}.