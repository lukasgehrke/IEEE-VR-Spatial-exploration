\subsection{Statistical Analyses}
\subsubsection{Predicting Presence using Participant Descriptives} First, we tested the accuracy of the predicted presence scores using general participant descriptors. We computed a least squares regression entering the IPQ presence (G1) scores as the dependent variable using R \cite{RFoundationforStatisticalComputing.2018}. For the predictors, we first computed individual averages over mazes and runs for movement velocity, exploration duration, number of wall touches, as well as sketch map accuracy. Further we added participants video game experience, sex, perspective taking and orientation ability and lastly the sense of direction into the regression model. For an explanation of each predictor see \cite{Gehrke2018}. Predictors that do not significantly add to the explanatory power of the regression model were localized using a step-wise model selection procedure based on Akaike's information criterion (AIC). The procedure was computed using 'stepAIC' of package 'MASS' \cite{Akaike1998a, Venables2002}. This procedure was selected to reduce over-fitting of the final reported model, meanwhile increasing the usability to other researchers through minimizing the number of included predictors. After the step-wise model selection, three predictors remained in the model predicting presence. Ultimately, the reduced model with three predictors was assessed in terms of its predictive accuracy. Therefore, a 5 fold cross-validation was computed to obtain a robust mean absolute error \cite{Mosteller1968, Furnkranz2011}. With 29 participants, each training fold consisted of either 23 or 24 participants with either 5 or 6 participants in the evaluated test fold. The R code and data used are available online\footnote{https://github.com/lukasgehrke/2019-IEEE-VR-Spatial-exploration-behavior-in-large-scale-VR-predicts-subjective-spatial-presence}.