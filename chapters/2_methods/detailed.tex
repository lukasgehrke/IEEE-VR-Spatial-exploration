\subsubsection{Parametric Mapping of Ongoing Movement Behavior} To enrich our analyses we strove for a greater spatial resolution of movement behavior. A general average of, e.g. time spent exploring, is limited in insight due to its spatial dependence. Therefore, we constructed spatial parametric maps of the movement behavior parameters. We mimicked our approach from established data analyses procedures in cognitive neuroscience and applied it to spatially resolved behavioral data \cite{Friston1994, Bridwell2018a}.

To keep our analyses to our reasonable scope we limited the investigated parameters to two: first, the time spent at each location as well as the number of wall touches elicited in close proximity. In short, this corresponds to where participants spent most of the time exploring the mazes as well as where they touched the wall most frequently. First each parameter was again average across within-subject repeated measures mazes and runs, see \cite{Gehrke2018}. Next, a 2D histogram with fixed edges to maintain equal resolution across participants was computed for each of the two parameters. For the map of the number of touches, the 2D histogram count of the location of the head was overwritten by the mean number of touches within the 2D histogram bin. Ultimately, and critically, a square Gaussian blur was applied to the histogram image to broaden the data and hence increase the overlap between participants. A sigma of 1.5 was chosen for the square filter kernel as it resulted in a good overlap across participants while maintaining spatial specificity.

To investigate the impact of the presence illusion on each parameter, we calculated a linear regression at each point of the map separately for the two parameters duration and number of touches. For each parameter and each pixel we calculated a linear regression across participants with the presence score as the lone predictor variable. The resulting parametric map shows the regression estimate, i.e. beta, of presence at each pixel. For statistical inference, we display a mask of significant p-values at an uncorrected $alpha < .05$. For interested researchers, the Matlab code used to construct the parametric maps is available online\footnote{https://github.com/lukasgehrke/mobi-3D-tools}.