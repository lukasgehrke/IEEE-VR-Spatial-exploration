\subsection{Procedure and Task}
\subsubsection{The Invisible Maze Task} Participants freely explored an interactive sparse invisible maze environment by walking and probing for virtual visual wall feedback with their hand, delivered by a virtual reality (VR) headset. Four different mazes (Fig. \ref{imt_task} B) were explored, each in three consecutive runs. Upon collision of the hand with an invisible wall, an illuminated white disc was displayed 30cm behind the collision point parallel to the invisible wall (Fig. \ref{imt_task} C). Due to the complexity of the technical details, please consult~\cite{Gehrke2018} for further details on the task, instrumentation and data collection. In summary, the task required participants to explore mazes to build a spatial representation of the maze layout. Doing the task, participants display a behavior that is comparable to explorative wall touches in the dark to find your way. We collected synchronized motion capture and behavioral events alongside high-density Electroencephalogry (EEG).