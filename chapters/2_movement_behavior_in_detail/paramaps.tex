% Address the question: What were the previous solutions to the stated problem?
\section{2 Related Work}
To this day, human development is explainable by the need to interact with a physical reality. Growing up, the human brain \textit{presumably} develops as a model of the latent hidden variables governing the observable behavior of cause and effect in the environment \cite{Friston2010}. Understanding cognition as a predictive process means brains constantly compare what is effectively happening in the world, with what was predicted to happen \cite{Clark2013}, for example inferring the trajectory of a thrown ball and successfully catching it.

\subsection{Natural behavior as a precursor of presence experience}
Successful immersion into virtual worlds relies on matching expectations that were substantiated in the non-virtual physical reality. Assuming that to experience presence in virtual environments equals treating what you perceive as a part of the reality you are currently in, many researchers have argued for an increase in ecological validity through VR experimentation across the cognitive sciences \cite{Bohil2011, Parsons2015, Parsons2017}. The assumption hence is that participants under the influence of successful VR illusions experience presence and therefore behave \textit{realistically} or with higher ecological validity. This work exhibits an approach to quantify such claims with spatial specificity in order to guide future design decisions in research and application. We point out, that this approach is scalable to investigations of spatially resolved behavioral, psychometric and bio-physiological parameters such as controller jerk, hit accuracy for example in video games or electroencephalographic (EEG) parameters \cite{Gehrke2019}.

\subsection{Presence experience impacts behavior}
Generally speaking many studies have investigated participants in VR illusions employing (A) emotionally charged stimulus material or (B) embodiment illusions. Considering (A), \cite{Diemer2015} provide a thorough overview of the intricate interplay between presence and reactions to emotionally charged stimulation in VR. The authors observed a consistently reported link between presence and the emotional experience in VR. They argue, that by varying degrees of arousal, presence impacts psychological as well as physiological responses. Considering (B), we highlight one relevant aspect from the rich literature on the effects of full-body embodiment into avatars \cite{Maister2015}. The proteus effect characterizes behavioral perturbations depending on avatar body attributes. Yee et al. showed that participants being immersed into an 'attractive' avatar moved closer into the interpersonal space of another person \cite{Yee2007}. Further, Banakou et al. demonstrated that within participants, the size of objects was overestimated while embodied into a child body compared to a non-embodied baseline \cite{Banakou2013}. Furthermore, one of the key VR illusions contributing to the subjective construct of presence is the place illusion which can be defined as the perception of oneself being present in a virtual place where one can act, react, and impact the surroundings \cite{Slater2009}. Therefore, self-location, sense of agency, and spatial awareness of the surroundings are strongly impacted by the place illusion and modify behavior \cite{Kilteni2012}. When one perceives oneself in control of ones own actions and observe action consequences in the virtual surroundings, spatial exploration behavior becomes a part of a learning process to adapt motor behavior to the surroundings as opposed to a random chain of actions executed by the user to explore, for instance, only the VR technology itself \cite{Tan2011}. Together, these examples illustrate that presence experience directly impacts cognition, the action-perception cycle.

\subsection{Benefits of studying spatially resolved behavior}
\textit{todo}
% Summary motivating our work:
% - what is the benefit of spatial resolution?
% find examples of mocap analysis collapsing data to the mean
% - cite studies with interesting findings but not spatially resolved, i.e. grand-averaged
%Problems of prior research in terms of the resolution of behavioral parameters investigate with respect to proteus effect or other. collapsing behavioral parameters to singular events


%%%%% writing resources
%The framework is derived from the neuroscientific methodology of parametric brain imaging, see for example \cite{Friston1994, Pernet2011}.
%mental representation of the surroundings and respectively the motor behavior.
%Novel immersive paradigms in the behavioral cognitive sciences as well as neurosciences subject participants to controlled, yet stimulus rich experiences in head-mounted VR. 
%Therefore, to be able to employ VR and claim ecological validity the level of presence must be accounted for. 