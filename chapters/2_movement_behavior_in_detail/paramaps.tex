% Address the question: What were the previous solutions to the stated problem?
\section{3 Related Work}
Traditionally, in order to address a given VR research challenge, specific aspects of the users behavior are discretized from the continuous motion capture stream. Aggregate data, such as averaged time spent in a specific area of interest or average and maximum speed moving along a specific trajectory are the simplest way to meet the researchers data processing requirements. Such a straight forward feature extraction scheme is widely established throughout diverse research communities as a top-down approach, driven by personal experience, expertise and interest. For example average time spent, as well as velocity aggregates have proven to be informative about crowd interaction~\cite{Nelson2019} and collaboration~\cite{Rios2018}. Increasing in the level of abstraction, feature extraction and classification guided by expert knowledge is useful for example in rehabilitation sciences with a motivation to derive informative features about rehabilitation progress, for example in reaching~\cite{DeLosReyes-Guzman2014} and gait applications~\cite{Taborri2016}. Extracting features of interest can be guided by expert knowledge, implemented as a tedious manual process, or automatically for example using  machine-learning~\cite{Butepage2017}. Recently, manual gait classification has been used in VR to investigate effects following gain factor modulations of avatar walking speeds~\cite{Abtahi2019}. Abtahi et. al. demonstrated that users adapt their walking behavior depending on, among other factors, the modulated gain factor, for example one step in the real world translating to traveling 10 meters in the VR~\cite{Abtahi2019}. In this work, we argue that a continuous assessment of movement behavior with 2D or 3D resolution may suit both researchers and developers addressing \textit{contextual} influences about their reported effects.

\subsection{Natural behavior as a precursor of presence experience}
To this day, human development is explainable by the need to interact with a physical reality. Growing up, the human brain \textit{presumably} develops as a model of the latent hidden variables, for example gravity, governing the observable behavior of cause and effect in the environment, for example an apple falling from a tree~\cite{Friston2010}. Understanding cognition as a predictive process holds that brains constantly compare what is effectively happening in the world with what was predicted to happen~\cite{Clark2013}, for example inferring the trajectory of a thrown ball and successfully catching it. 

Successful immersion into virtual worlds relies on matching expectations that were substantiated in the non-virtual physical reality. Assuming that to experience presence in virtual environments equals treating what you perceive as a part of the reality you are currently in, many researchers have argued for an increase in ecological validity through VR experimentation. The assumption is that participants under the influence of successful VR illusions experience presence and therefore behave \textit{realistically} or with higher ecological validity~\cite{Tarr2002, Bohil2011, Parsons2015, Parsons2017}. This work exhibits an approach to quantify such claims with high spatial resolution in order to guide future design decisions in research and application. We point out, that this approach is scalable to investigations of spatially resolved behavioral, psychometric and bio-physiological parameters such as controller jerk, hit accuracy for example in video games or electroencephalographic (EEG) parameters.

\subsection{Presence experience impacts behavior}
Generally speaking many studies have investigated participants under VR illusions employing (A) emotionally charged stimulus material or (B) embodiment illusions. Considering (A),~\cite{Diemer2015} provide a thorough overview of the intricate interplay between presence and reactions to emotionally charged stimulation in VR. The authors observed a consistent link between presence and the emotional experience in VR. They argue, that by varying degrees of arousal, presence impacts psychological as well as physiological responses. Considering (B), Maister et al. highlight one relevant aspect from the rich literature on the effects of full-body embodiment into avatars~\cite{Maister2015}. The proteus effect characterizes behavioral perturbations depending on avatar body attributes. Yee et al. showed that participants being immersed into an `attractive' avatar moved closer into the interpersonal space of another person~\cite{Yee2007}. Further, Banakou et al. demonstrated that for participants that were embodied into an avatar with a children's body, the size of objects was overestimated as compared to a non-embodied baseline~\cite{Banakou2013}.

One of the key VR illusions contributing to the subjective construct of presence is the place illusion which can be defined as the perception of oneself being present in a virtual place where one can act, react, and impact the surroundings~\cite{Slater2009}. Therefore, self-location, sense of agency, and spatial awareness of the surroundings are strongly impacted by the place illusion and modify behavior~\cite{Kilteni2012}. When one perceives oneself in control of ones own actions and observed action consequences in the virtual surroundings, spatial exploration behavior becomes a part of a learning process to adapt motor behavior to the surroundings as opposed to a random chain of actions executed by the user to explore, for instance, only the VR technology itself~\cite{Tan2011}. Together, these examples illustrate that presence experience directly impacts cognition, the predictive action-perception cycle.