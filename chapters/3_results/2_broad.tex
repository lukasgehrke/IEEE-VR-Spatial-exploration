\subsection{Video Game Experience, Gender and Perspective Taking predict experienced Presence} Running a step-wise model selection resulted in three predictors being kept, explaining 53,4\% of the variation in experienced presence ($F_{(3,25)}=11.69, p < .001,$ adjusted $R^2=.534$). Participants' predicted presence was equal to $8.15 - 1.2 (Video Game Experience) + 2.61 (SEX) - 0.02 (PTSOT)$ where sex was dummy-coded as 1 = Male, 2 = Female, increasing video game experience was coded with higher scores and decreasing perspective taking ability with higher scores. Video game experience ($t_{(25)}=-4.7, p<.001$), sex ($t_{(25)}=5.6, p<.001$) as well as perspective taking ability ($t_{(25)}=-2.52, p<.05$) were significant predictors of presence.

Training the three predictor model above for each of five different folds of the data and evaluating its performance on the held-out fold yielded a combined average .76 mean absolute error. Hence, using video game experience, sex as well as perspective taking ability we were able to predict experienced presence to within three-quarters of a point accuracy.