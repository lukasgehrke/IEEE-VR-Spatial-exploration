\subsubsection{Predicting Presence using Participant Descriptives} 
To follow up our spatially-resolved analyses and situate our findings more generally, we zoomed out and set out to predict presence scores using aggregate data and participant descriptors. We computed a least squares regression entering the IPQ presence (G1) scores as the dependent variable using R~\cite{RFoundationforStatisticalComputing.2018}. To accentuate the effect of aggregating motion capture data, we included aggregates of the exploration duration in the model. We aggregated over mazes and runs. Further we added participants video game experience, biological sex and perspective taking and orientation ability into the regression model. For a detailed explanation of each predictor consult the data source~\cite{Gehrke2018}. 

Predictors that did not significantly add to the explanatory power of the regression model were localized using a step-wise model selection procedure based on Akaike's information criterion (AIC). The procedure was computed using `stepAIC' of package `MASS'~\cite{Akaike1998a, Venables2002}. This data-driven procedure was selected to exclude the likely problem of over-fitting for the final reported model and to increase the usability of the approach by minimizing the number of included predictors. To assess the predictive accuracy of the reduced model, a 5 fold cross-validation was computed to obtain a robust mean absolute error~\cite{Mosteller1968, Furnkranz2011}. With 29 participants, each training fold consisted of either 23 or 24 participants with either 5 or 6 participants in the evaluated test fold. The R code is available online online\footnote{https://github.com/lukasgehrke/2019-IEEE-VR-Spatial-exploration-behavior-in-large-scale-VR-predicts-subjective-spatial-presence} and the data can be made available upon request.

%%%%% old
%To follow up our spatially-resolved analyses and situate our findings more generally, we zoomed out and set out to predict presence scores using aggregate data and participant descriptors. We computed a least squares regression entering the IPQ presence (G1) scores as the dependent variable using R~\cite{RFoundationforStatisticalComputing.2018}. To accentuate the effect of aggregating motion capture data we, we included aggregates of movement velocity, time spent exploration or exploration duration. Additionally, we included the average number participants touched a wall with their hand as well as the average accuracy of a sketch map participants had to draw following each exploration. The last metric alludes to an accurate mental spatial representation, which was specifically queried with the \textit{invisible maze task}. We aggregated the measures by computing individual averages over mazes and runs. Further we added participants video game experience, sex, perspective taking and orientation ability and lastly the sense of direction into the regression model. For a detailed explanation of each predictor consult~\cite{Gehrke2018}. Predictors that do not significantly add to the explanatory power of the regression model were localized using a step-wise model selection procedure based on Akaike's information criterion (AIC). The procedure was computed using `stepAIC' of package `MASS'~\cite{Akaike1998a, Venables2002}. This data-driven procedure was selected to exclude the likely problem of over-fitting for the final reported model and to increase the usability of the approach by minimizing the number of included predictors. After the step-wise model selection, three predictors remained in the model predicting presence. Ultimately, the reduced model with three predictors was assessed in terms of its predictive accuracy. Therefore, a 5 fold cross-validation was computed to obtain a robust mean absolute error~\cite{Mosteller1968, Furnkranz2011}. With 29 participants, each training fold consisted of either 23 or 24 participants with either 5 or 6 participants in the evaluated test fold. The R code and data used are available online\footnote{anonymized}.