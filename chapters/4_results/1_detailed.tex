\subsection{Exploration behavior expressed as a function of presence} 
First, averaging spatially-resolved exploration time across participants revealed that participants spent more time in dead-ends as well as in areas including corners. The effect was observed across all mazes (see red colors in figure \ref{head_loc_mean}). Conversely, less time spent and faster movement was observed for pixels of straight segments of the mazes. As expected, participants also spent more time at the beginning of the trial, taking a moment before starting to move. Together, these findings confirmed our intuitive expectations of spatial exploration behavior. 

Second, we investigated the effect of presence on time spent exploring mass-univariately for all locations in all mazes. With increasing presence scores, time spent at the center of the straight paths increased, for example in maze `I' at [3.5, 1.5] $beta=.027, t_{(28)}=2.76, p=.01, R^2=.23$ and `Z' at [1.8, 10] $beta=.039, t_{(28)}=3.34, p<.01, R^2=.34$, see figure \ref{presence_head_loc} A. Further, increasing presence scores correlated with more time spent at the center of the paths in corners, for example in maze `L' at [-1, 3.5] $beta=.037, t_{(28)}=2.33, p=.03, R^2=.19$ and `U' at [0.5, 17.5] $beta=.034, t_{(28)}=2.44, p=.02, R^2=.19$. Conversely, closer to the maze boundaries, we observed a negative effect with increasing presence scores correlating with less time spent in these areas, specifically in the most challenging mazes `Z' at [2, 11.5] $beta=-.073, t_{(28)}=-4.16, p<.01, R^2=.40$ and `U' at [0.5, 17] $beta=-.143, t_{(28)}=-4.05 p<.01, R^2=.40$. Taken together, with an increase in experienced presence, participants spent more time firmly located at the center of the path, specifically along the straight segments in the `I' and `Z' maze as well in navigationally relevant corners of mazes `L', `Z' and `U'. Further underlining this observation, higher reported presence scores negatively correlated with the time spent close or even colliding with the maze walls.