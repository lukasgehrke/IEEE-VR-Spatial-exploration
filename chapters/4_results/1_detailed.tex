\subsection{Zooming in: Exploration behavior expressed as a function of presence} 
First, averaging spatially-resolved exploration time across participants revealed that participants spent more time in dead-ends as well as in the corners. The effect was congruently observed across all mazes, see hot colors in figure \ref{head_loc_mean}. Conversely, less time was spent when located in the straight segments of each maze, i.e. participants moved faster in those segments. As expected participants also spent more time at the beginning of the trial, taking a moment to start moving. Together, these findings confirmed our intuitive expectations of the exploration behavior. 

Second, we investigated the effect of presence on time spent exploring mass-univariately for all locations in all mazes. With increasing presence, time spent at the center of the straight paths increased, for example in maze `I' at [3.5, 1.5] $beta=.027, t_{(28)}=2.76, p=.01, R^2=.23$ and `Z' at [1.8, 10] $beta=.039, t_{(28)}=3.34, p<.01, R^2=.34$, see figure \ref{presence_head_loc} A. Further, increasing presence correlated with more time spent at the center of the paths in the corners, for example in maze `L' at [-1, 3.5] $beta=.037, t_{(28)}=2.33, p=.03, R^2=.19$ and `U' at [0.5, 17.5] $beta=.034, t_{(28)}=2.44, p=.02, R^2=.19$. Conversely, closer to the maze boundaries we observed a negative effect with increasing presence score correlating with less time spent, specifically in the most challenging mazes `Z' at [2, 11.5] $beta=-.073, t_{(28)}=-4.16, p<.01, R^2=.40$ and `U' at [0.5, 17] $beta=-.143, t_{(28)}=-4.05 p<.01, R^2=.40$. Taken together, with an increase in experienced presence, participants spent more time firmly located at the center of the path, specifically along the straight segments in the `I' and `Z' maze as well in navigationally relevant corners of mazes `L;, `Z' and `U'. Further underlining this observation, higher reported presence negatively correlated with time spent close or even crashing into the maze walls.

%%%%% writing ressources, old analyses
%The effect of presence on the number of spatially resolved wall touches showed less spatial specificity. Participants with higher reported presence had overall more wall touches than participants with a lower score. Presence did not significantly impact the spatial distribution of wall touches during exploration. We did observe a positive effect of presence on the number of wall touches along inside corner and dead-ends of mazes 'L' and 'U'. Higher experienced presence coincided with an increase in the number of touches along those corners and dead-ends, see figure \ref{results_touches}.