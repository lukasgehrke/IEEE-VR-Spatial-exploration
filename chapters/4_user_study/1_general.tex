\section{4 user study}
In the present work, we address whether the level of experienced presence affects 'realistic' ongoing motor behavior, i.e. spatial exploration behavior in a large VR mimicking a real-world exploration situation such as finding your way in complete darkness. Here, we investigated the impact of experienced presence on spatial exploration behavior in the invisible maze task~\cite{Gehrke2018}. To address our motivation, we conducted the following two-step analysis. In a first step, we built a linear model to predict experienced presence given per participant descriptors. We were primarily interested to see how accurate presence can be predicted from broad knowledge of participant movement behavior. Additionally we considered a multitude of participant descriptors to derive a useful model. We reduced the model to a minimum of useful predictors so other researchers may easily reproduce our findings. In other words, we addressed how accurate subjectively reported presence may be predicted given a number of per participant descriptors.
In a second step, we increased the resolution of our analyses to specifically investigate ongoing motor behavior. Here, we analysed in detail whether participants movement behavior differed as a function of experienced presence and where it did so. To this end, we conducted mass-univariate pixel-by-pixel modeling of experienced presence on duration spent in a certain location and the number of wall touches elicited there, the simple where and what of participants actions.

\subsection{Participants, Procedure, Task and Setup} Thirty-two healthy participants (aged 21--45 years, 14 men) took part in the experiment. All participants gave written informed consent to participation and the experimental protocol was approved by the local ethics committee (protocol: GR\_08\_20170428). Three participants were excluded from data analysis due to incomplete data or difficulties in complying with the task requirements.

%Here, we propose a highly accurate subjective presence metric based on individual differences in video game experience, sex, and spatial perspective taking abilities. Furthermore, we show that the level of presence impacts free spatial exploration behavior in a large scale VR during a spatial navigation paradigm. Finally, using a methodological framework developed for the cognitive neuroscience, we showcase a powerful analysis framework to investigate ongoing behavior in cognitive paradigms in greater detail.

%Finally, using a methodological framework developed for the cognitive neuroscience, we showcase a powerful analysis framework to investigate ongoing behavior in cognitive paradigms in greater detail.

\subsubsection{The Invisible Maze Task} Participants freely explored a sparse invisible maze environment interactively by walking and probing for visual feedback when touching the virtual wall with their hand. All stimuli were presented using an Oculus Rift DK2 VR headset.
% explain benefit of DK2 (Weight, modifiability) and of tracking system (accurate in large space)
Upon collision of the hand with an invisible wall, an illuminated white disc was displayed 30 cm behind the collision point parallel to the invisible wall (Fig. \ref{imt_task} C) (see ~\cite{Gehrke2018} for further details on the task, instrumentation, and data collection). In summary, the task required participants to explore mazes to build a spatial representation of the maze layout. Four different mazes (Fig. \ref{imt_task} B) were explored each in three consecutive runs. Doing the task, participants display a behavior that is comparable to explorative wall touches in the dark to find your way. We collected synchronized motion capture and behavioral events alongside high-density Electroencephalogry (EEG).

\subsubsection{Assessing presence} To assess experienced presence, we administered the igroup presence questionnaire after participants explored four different mazes repeatedly for about one hour \cite{Schubert2003}. For the reported analyses we considered only the first item of the questionnaire, the subjective presence measure (G1) is representing the sense of being in a place, i.e. 'In the computer generated world I had a sense of "being there' rated from 'not at all' to 'very much' \cite{Schubert2003, Slater1993}.