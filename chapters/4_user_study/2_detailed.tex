\subsection{Statistical Analyses}
To enrich our analyses we strove for a greater spatial resolution of movement behavior. A general average of, e.g. time spent exploring the mazes, provides only limited insights due to its spatial dependence. Some participants may spent more time in the corners but walk faster along straights segments. Without a two-dimensional resolution, this would average out to the same as participants with a constant walking speed. Therefore, we constructed spatial parametric maps of the movement behavior parameters. We mimicked our approach from established data analyses procedures in cognitive neuroscience and applied it to spatially resolved behavioral data \cite{Friston1994, Bridwell2018a}.

\subsubsection{Parametric Mapping of Ongoing Movement Behavior}
To reduce the complexity of the analysis, we limited the parameters to two parameters of interest: first, the time spent at each location as well as the number of wall touches elicited in close proximity. In short, this corresponds to where participants spent most of the time exploring the mazes as well as where they were located when touching the walls most frequently. First, we computed individual averages over mazes and runs. Next, a 2D histogram with fixed edges to maintain equal resolution across participants was computed for each of the two parameters. For the map displaying the number of touches, the 2D histogram count of the location of the participants' head was overwritten by the mean number of touches within the 2D histogram bin. Ultimately, a 2D (square sized) Gaussian blur was applied to the histogram image to broaden the data and to increase the overlap between participants. A sigma of 1.5 was chosen for the 2D filter kernel as it resulted in a good overlap across participants while maintaining spatial specificity.
% todo data was subsampled to 1 sample per second

To investigate the impact of experienced presence on each parameter, we calculated a linear regression at each point of the map separately for the two parameters duration and number of touches. For each parameter and each pixel we calculated a linear regression across participants with the presence score as predictor variable. As not all participants had a data point at each pixel in the map, we only computed a regression for all pixels with more than 12 participants.

The resulting parametric map shows the regression estimate, i.e. beta, at each pixel. To address multiple comparison issues with the mass-univariate regression approach, we applied threshold free cluster enhancement to the results F and t maps % cite smith nichols, pernet
We thresholded the tfce transform of the true result maps at the 95th percentile (alpha $< .05$) of the max tfce distribution of the bootstraps. A contour mask of significant betas across subjects is displayed. The Matlab code used to construct the parametric maps is available online\footnote{https://github.com/lukasgehrke/mobi-3D-tools}.