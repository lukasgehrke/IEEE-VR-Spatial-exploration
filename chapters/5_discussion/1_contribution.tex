\section{6 Contribution, Benefits \& Limitations}
With this paper, we motivate developers and researchers to use parametric mapping using the inherit motion capture capabilities of VR technology for data-driven understanding of user behavior and, ultimately, experience. We introduced parametric maps, expressing VR exploration behavior as a function of experienced presence as an exemplary predictor variable related to the subjective user experience. 

In the user study we showed that increasing subjective presence scores coincided with an increased time spent at the center of the paths' through our simple \textit{invisible} mazes. In turn, this effect translates to participants with high presence scores spending less time closer to the walls, or crashing into them, presumably relating to high spatial awareness. The spatial specificity of our reported effect accentuates that aggregate behavioral metrics, such as average speed or time spent across one maze exploration, may miss critical information. Here a finer resolution has proven to be beneficial.

Our analyses exposed that high presence scores coincided with more time spent in the corners, further underlining the hypotheses of increasing spatial awareness with increasing presence. Sense of embodiment has frequently been included as a key component in psychological models of presence experience with a sense of self-location as one emergence of embodiment~\cite{Kilteni2012}. This sense of a virtual self is a logical prerequisite for spatial anchoring and reduced cognitive effort when imagining a third-person allocentric perspective of space. Here additional mental transformations potentially disrupt the presence experience by violating the \textit{natural} experience of the predictive brain~\cite{Gonzalez-Franco2017}. In consequence, solving spatial task, for example requiring perspective taking, depend on individual abilities and preferences for specific spatial strategies but appear to be `easier' with a congruent sense of embodiment~\cite{Pan2018, Gramann2005, Cognition2016}.

Using participants individual characteristics, we were able to classify presence scores with an average deviation of $/pm 0.75$ from the true reported score. Employing a data-driven model selection procedure determined that apart from video game experience, biological sex as well as perspective taking ability, no other predictor significantly contributed to the linear classification scheme. Interestingly, in our experiment increasing video game experience negatively impacted presence scores, conflicting with previous findings~\cite{Lachlan}. We hypothesize that this could be explained by the overly sparse visuals of the virtual environment in the \textit{invisible maze task}. Participants with significant video gaming experience might perceive the world as too artificial. With regards to the impact of biological sex, we contribute one more example of contradictory results in the literature~\cite{Coluccia2004}. Based on our outcome, however, we cannot argue that the underlying differences in the task were due to a spatial component or rather the interaction with an unknown sparse virtual environment and refer to a detailed discussion on the topic~\cite{Felnhofer}. Ultimately, perspective taking ability had a limited influence predicting subjective presence. Increasing perspective taking scores, referring to an angular error in a mental triangulation task, negatively impacted the subjective feeling of presence. We hypothesize that immersion, as one precursor for the experience of presence may arise earlier in individuals that feel oriented and that are aware of their spatial surroundings. As such, spatial awareness is useful when solving tasks in unfamiliar environments~\cite{Slater2018}.

\section{7 Conclusion}
With this work we motivate continuous, spatially resolved, analyses to understand individual characteristics explaining exploration behavior in VR, ultimately explaining user experience. However, user experience, for example presence experience, is subjective and traditionally assessed via questionnaires following the experience to be evaluated. Such an approach exhibits several advantages. First, a continuous assessment is desirable over a discrete sampling after the VR experience and second, using active and continuous behavioral parameters without interrupting the ongoing experience to administer the questionnaire allows for a non-distorted assessment of presence.

\subsection{Towards an unobtrusive, spatially resolved understanding of user experience}
With this work we showed that observing finely resolved overt behavior may provide hints at the subjective user experience, such as presence experience. Using physiological methods holds the potential to overcome indirect measurement of behavior allowing to directly assess the physiological source that realizes the subjective experience, i.e. the brain~\cite{Gehrke2019, Singh2018, Si-mohammed2020}. Gehrke et al. demonstrated that neural responses associated with mental error-processing following visuo-haptic mismatches in VR can be detected using EEG~\cite{Gehrke2019}. Recently, Si-Mohammed et al. were able to use this EEG signature in a brain-computer interface classification scheme exhibiting high classification accuracy in a similar paradigm exhibiting graphical glitches in VR. These promising findings, hint at future applications in room-scale scenarios, picking up scenarios where a loss of immersion occurs in near real-time. Here, additional wearable sensors can supplement brain recordings to, for example, measure affective responses~\cite{Marin-Morales2018}.

To sum up, we propose a method to investigate user experience in VR by using a spatially resolved analyses approach of ongoing behavior, providing designers and researchers a tool to guide future design decisions. Investigating how individual characteristics and the current spatial context influences user experience provides the opportunity for inclusive design decisions driving VR technology acceptance across the broad public. Ultimately, we believe our approach can further be nurtured by unobtrusive, continuous (neuro-)physiological measurements for a multi-dimensional assessment of user experience.