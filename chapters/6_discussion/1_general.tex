\section{6 discussion}
Our work reports two relevant contributions. First it confirms established results regarding contributing factors to participants propensity to report high or low presence score after a virtual reality experience.

Besides an interest in predicting the level of experienced presence in real-time with neurophysiological metrics, investigating whether there are observable effect in overt behavior is of equal interest and importance to the ecological validity argument.

% cite results where sex and video game experience contribute to presence

presence and accuracy of motor behavior, problem because non-continuous metric, cite myself, moderated by learning/difficulty, clustering approaches

presence is best predicted by video game experience and sex (there is evidence of sex and videogame influence in slater work etc.). Interestingly, in our experiment video game experience negatively impacts presence reported on the general item of the IPQ. This may be due to the overly simplistic visuals of the virtual world. Participants with significant video gaming experience might perceive the world as too artificial.

explore exit interviews and use in discussion!

%\subsection{Disentangling Presence}
Sense of ownership, sense of agency etc.

% put first image of heatmaps of location and then hint at using mass-univariate regression on each point to increase the resolution of the investigative lens