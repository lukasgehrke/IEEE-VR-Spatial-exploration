% Address the question: What were the previous solutions to the stated problem?
\section{3 Related Work}
% Summarize what was previously done: physiological heatmaps, see, feel, move
Traditionally, in order to address a given VR research challenge, specific aspects of the users behavior are discretized from the continuous motion capture stream. Aggregate data, such as averaged time spent in a specific area of interest or average and maximum speed moving along a specific trajectory are the simplest way to meet the researchers data processing requirements. Such a straight forward feature extraction scheme is widely established throughout diverse research communities as a top-down approach, driven by personal experience, expertise and interest. For example average time spent, as well as velocity aggregates have proven to be informative about crowd interaction~\cite{Nelson2019} and collaboration~\cite{Rios2018}. Increasing in the level of abstraction, feature extraction and classification guided by expert knowledge is useful for example in rehabilitation sciences with a motivation to derive informative features about rehabilitation progress, for example in reaching~\cite{DeLosReyes-Guzman2014} and gait applications~\cite{Taborri2016}. Extracting features of interest can be guided by expert knowledge, implemented as a tedious manual process, or automatically for example using  machine-learning~\cite{Butepage2017}. Recently, manual gait classification has been used in VR to investigate effects following gain factor modulations of avatar walking speeds~\cite{Abtahi2019}. Abtahi et. al. demonstrated that users adapt their walking behavior depending on, among other factors, the modulated gain factor, for example one step in the real world translating to traveling 10 meters in the VR~\cite{Abtahi2019}. In this work, we argue that a continuous assessment of movement behavior with 2D or 3D resolution may suit both researchers and developers addressing \textit{contextual} influences about their reported effects.

% What is still missing and how does our approach overcome this?
% -> then give two examples on how to apply this approach