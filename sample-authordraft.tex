%%
%% This is file `sample-authordraft.tex',
%% generated with the docstrip utility.
%%
%% The original source files were:
%%
%% samples.dtx  (with options: `authordraft')
%% 
%% IMPORTANT NOTICE:
%% 
%% For the copyright see the source file.
%% 
%% Any modified versions of this file must be renamed
%% with new filenames distinct from sample-authordraft.tex.
%% 
%% For distribution of the original source see the terms
%% for copying and modification in the file samples.dtx.
%% 
%% This generated file may be distributed as long as the
%% original source files, as listed above, are part of the
%% same distribution. (The sources need not necessarily be
%% in the same archive or directory.)
%%
%% The first command in your LaTeX source must be the \documentclass command.
% \documentclass[sigconf,manuscript,review,anonymous]{acmart}
\documentclass[sigconf]{acmart}

%%
%% \BibTeX command to typeset BibTeX logo in the docs
\AtBeginDocument{%
  \providecommand\BibTeX{{%
    \normalfont B\kern-0.5em{\scshape i\kern-0.25em b}\kern-0.8em\TeX}}}

%% Rights management information.  This information is sent to you
%% when you complete the rights form.  These commands have SAMPLE
%% values in them; it is your responsibility as an author to replace
%% the commands and values with those provided to you when you
%% complete the rights form.
\setcopyright{acmcopyright}
\copyrightyear{2021}
\acmYear{2021}
\acmDOI{10.1145/1122445.1122456}

%% These commands are for a PROCEEDINGS abstract or paper.
\acmConference[]{}{}{}
\acmBooktitle{}
\acmPrice{15.00}
\acmISBN{978-1-4503-XXXX-X/18/06}


%%
%% Submission ID.
%% Use this when submitting an article to a sponsored event. You'll
%% receive a unique submission ID from the organizers
%% of the event, and this ID should be used as the parameter to this command.
%%\acmSubmissionID{123-A56-BU3}

%%
%% The majority of ACM publications use numbered citations and
%% references.  The command \citestyle{authoryear} switches to the
%% "author year" style.
%%
%% If you are preparing content for an event
%% sponsored by ACM SIGGRAPH, you must use the "author year" style of
%% citations and references.
%% Uncommenting
%% the next command will enable that style.
%%\citestyle{acmauthoryear}

%%
%% end of the preamble, start of the body of the document source.
\begin{document}

%%
%% The "title" command has an optional parameter,
%% allowing the author to define a "short title" to be used in page headers.
\title{Exposing Movement Correlates of Presence Experience in Virtual Reality using Parametric Maps}

%%
%% The "author" command and its associated commands are used to define
%% the authors and their affiliations.
%% Of note is the shared affiliation of the first two authors, and the
%% "authornote" and "authornotemark" commands
%% used to denote shared contribution to the research.

% \author{Ben Trovato}
% \authornote{Both authors contributed equally to this research.}
% \email{trovato@corporation.com}
% \orcid{1234-5678-9012}
% \author{G.K.M. Tobin}
% \authornotemark[1]
% \email{webmaster@marysville-ohio.com}
% \affiliation{%
%   \institution{Institute for Clarity in Documentation}
%   \streetaddress{P.O. Box 1212}
%   \city{Dublin}
%   \state{Ohio}
%   \postcode{43017-6221}
% }

% \author{Lars Th{\o}rv{\"a}ld}
% \affiliation{%
%   \institution{The Th{\o}rv{\"a}ld Group}
%   \streetaddress{1 Th{\o}rv{\"a}ld Circle}
%   \city{Hekla}
%   \country{Iceland}}
% \email{larst@affiliation.org}

% \author{Valerie B\'eranger}
% \affiliation{%
%   \institution{Inria Paris-Rocquencourt}
%   \city{Rocquencourt}
%   \country{France}
% }

% \author{Aparna Patel}
% \affiliation{%
%  \institution{Rajiv Gandhi University}
%  \streetaddress{Rono-Hills}
%  \city{Doimukh}
%  \state{Arunachal Pradesh}
%  \country{India}}

% \author{Huifen Chan}
% \affiliation{%
%   \institution{Tsinghua University}
%   \streetaddress{30 Shuangqing Rd}
%   \city{Haidian Qu}
%   \state{Beijing Shi}
%   \country{China}}

% \author{Charles Palmer}
% \affiliation{%
%   \institution{Palmer Research Laboratories}
%   \streetaddress{8600 Datapoint Drive}
%   \city{San Antonio}
%   \state{Texas}
%   \postcode{78229}}
% \email{cpalmer@prl.com}

% \author{John Smith}
% \affiliation{\institution{The Th{\o}rv{\"a}ld Group}}
% \email{jsmith@affiliation.org}

% \author{Julius P. Kumquat}
% \affiliation{\institution{The Kumquat Consortium}}
% \email{jpkumquat@consortium.net}

%%
%% By default, the full list of authors will be used in the page
%% headers. Often, this list is too long, and will overlap
%% other information printed in the page headers. This command allows
%% the author to define a more concise list
%% of authors' names for this purpose.
% \renewcommand{\shortauthors}{Trovato and Tobin, et al.}
\renewcommand{\shortauthors}{}

%%
%% The abstract is a short summary of the work to be presented in the
%% article.
% \begin{abstract}
%   A clear and well-documented \LaTeX\ document is presented as an
%   article formatted for publication by ACM in a conference proceedings
%   or journal publication. Based on the ``acmart'' document class, this
%   article presents and explains many of the common variations, as well
%   as many of the formatting elements an author may use in the
%   preparation of the documentation of their work.
% \end{abstract}
% must be between 150 and 250, so aiming for 200!
\begin{abstract}
\textit{this is from an older version and needs to be reworked to fit the current narrative, just a placeholder for the abstract length to format where figures appear}
% todo add more on parametric maps here in the beginning
We show that the level of presence impacts free spatial exploration behavior in a large scale VR navigation paradigm. Here, we investigate the impact of an established presence metric on ongoing motor behavior demonstrating an analysis framework with a high spatial resolution. We observed participants with higher presence exhibit presumably higher spatial awareness by spending more time at navigationally relevant segments. Ultimately, we link presence to individual differences in video game experience, sex, and spatial perspective taking abilities, confirming that user characteristics are a defining part of the presence constructy spending more time at navigationally relevant segments. Ultimately, we link presence to individual differences in video game experience, sex, and spatial perspective taking abilities, confirming that user characteristics are a defining part of the presence construct.
%Video game experience, biological sex as well as spatial perspective taking skills proved to be significant predictors of a rich presence experience and were able to predict presence to within 3/4 of a point accuracy on a standard questionnaire scale. 

\end{abstract}
% in the end use hemingway app (http://www.hemingwayapp.com) to increase readability
% Word count is 190 and must be 150

%old
%The use of head-mounted virtual reality (VR) to induce presence in a computer simulated world has proven to significantly increase the ecological validity of this medium. In VR, illusions of various kinds (place illusion, plausibility illusion, etc.) occur at the same time for the user to feel present. Most prominently, the embodiment illusion has proven to elicit 'realistic' behavioral as well as physiological responses, when a strong emotional stimulus such as virtual hurting of an embodied rubber hand is provided. Yet, to be able to employ VR and claim ecological validity for less emotional stimuli, the level of presence must be accounted for. .

%%
%% The code below is generated by the tool at http://dl.acm.org/ccs.cfm.
%% Please copy and paste the code instead of the example below.
%%
\begin{CCSXML}
<ccs2012>
   <concept>
       <concept_id>10003120.10003121.10003122.10003334</concept_id>
       <concept_desc>Human-centered computing~User studies</concept_desc>
       <concept_significance>500</concept_significance>
       </concept>
   <concept>
       <concept_id>10003120.10003121.10003124.10010866</concept_id>
       <concept_desc>Human-centered computing~Virtual reality</concept_desc>
       <concept_significance>300</concept_significance>
       </concept>
   <concept>
       <concept_id>10003120.10003121.10011748</concept_id>
       <concept_desc>Human-centered computing~Empirical studies in HCI</concept_desc>
       <concept_significance>100</concept_significance>
       </concept>
 </ccs2012>
\end{CCSXML}

\ccsdesc[500]{Human-centered computing~User studies}
\ccsdesc[300]{Human-centered computing~Virtual reality}
\ccsdesc[100]{Human-centered computing~Empirical studies in HCI}

%%
%% Keywords. The author(s) should pick words that accurately describe
%% the work being presented. Separate the keywords with commas.
\keywords{Virtual reality; Motion capture; Statistical parametric mapping; Presence;}

%% A "teaser" image appears between the author and affiliation
%% information and the body of the document, and typically spans the
%% page.
\begin{teaserfigure}
  \includegraphics[width=\textwidth]{figures/fig1_3d_new.pdf}
  \caption{We propose parametric maps to study user behavior and experience and guide future design of room-scale VR applications. \textbf{A,} Top-down view on the participant equipped with a wireless VR headset and backpack PC at the start of a `Z' maze. Motion capture of exploration paths (solid pink line) was spatially blurred using a Gaussian filter to obtain images/maps in B suited for across participants analyses of moderating variables. \textbf{B,} We constructed parametric maps of where participants spent time exploring the mazes as a function of their experienced presence. \textbf{C,} We found that with increasing presence, participants were more likely to stay in the center of the paths as well as in segments \textit{presumably} critical for navigational success. Significant pixels at $p<0.05$ are masked with a dotted line}
  \Description{A: Movement Trajectory of a participant exploring a virtual maze environment. B: Trajectoyies of many participants with their presence questionnaire score given after exploration. C: Regression analyses using presence scores as a predictor variable to express where participants spent time exploring the virtual environment}
  \label{fig:methods}
\end{teaserfigure}

%%
%% This command processes the author and affiliation and title
%% information and builds the first part of the formatted document.
\maketitle

% 1 Introduction
% What problem are you solving, or opportunity are you seizing? 
\section{1 intro}
Especially in times of a global pandemic, increasing the (psychological) depth of people's remote connections is desirable across the board. Room-scale VR significantly increases the users immersion by, first and foremost, allowing free movements, simulating the sensory \textit{natural} experience in the real world. Further, synchronized motion capture allows avatar movements to be rendered equal to the own bodies movements. In VR, illusions of various kinds (place illusion, plausibility illusion, etc.) occur at the same time for the user to feel present. Depending on the effectiveness of the employed illusions and whether multiple illusions work in congruence, participants experience feeling present in VR \cite{Gonzalez-Franco2017, Kilteni2012}. Most prominently, the embodiment illusion, or body ownership transfer, has proven to elicit \textit{realistic} behavioral as well as physiological responses, when a strong emotional stimulus such as virtual hurting of an embodied rubber hand is provided. Such illusions enrich and/or modify participants subjective experience in VR. Together, free movement, \textit{realisitic} avatar rendering and contextual components of the VR experience coincide at the same time for the user to feel present. Designing immersive experiences for room-scale VR aims at facilitating the emergence of presence experience, dissolving the feeling of connectedness to the real body, ultimately providing the foundation for genuinely connected (social) experiences. 

\section{Zooming in: presence experience and its impact on movement behavior}
In order to scale immersive VR technology to a broader public with use cases ranging from remote office work to entertainment, inclusive design principles are of key importance to successfully designing presence experience across the user base. Individual differences, for example the proclivity to move throughout virtual worlds, significantly challenge designing for presence experience thereby challenging acceptance of VR technology in general \cite{Sagnier2020}. Here, designers and developers would benefit from a rich understanding of user behavior, being able to directly query the impact of certain characteristics of their, unintendedly, neglected user base. Specifically in room-scale VR applications, leveraging inherent motion capture provides the opportunity for data-driven understanding of how individual proclivities explain user experience with spatial specificity. However, typical data reports usually consider aggregate data missing the opportunity to spatially resolve effects of interest.

\begin{figure*}[!ht]
\centering
  \includegraphics[width=\linewidth]{figures/fig1.pdf}
  %\vspace{-15pt}
  \caption{We propose parametric maps to guide future design decisions of room scale VR applications addressing a broader public. In our study participants explored \textit{invisible} mazes probing hidden walls for visual guidance. \textbf{A:} Motion capture of exploration paths was spatially blurred for across participants analyses of moderating variables. \textbf{B:} We administered experience presence and constructed parametric maps of where participants spent time exploring the mazes as a function of their experienced presence. \textbf{C:} Validating our proposal, we found that with increasing presence, participants were more likely to stay in the center of the paths as well as in segments critical for navigational success.}~\label{fig:methods}
\end{figure*}

We propose using \textit{GLM}, general linear model, in a mass-univariate application across all pixels of a given room-scale VR space. The approach is scalable to many models in the \textit{GLM family} therefore providing a framework with significant flexibility. Further, the analyses scheme is easily extended to 3D space, for example to investigate individual differences in hand reaching scenarios. In this work, we employ the proposed analyses scheme to exhibit how differences in subjectively reported experienced presence impacts participants movement profiles in a \textit{beyond} room-scale VR spatial exploration task. We guide the reader in a step-by-step fashion through the pre-processing, model fitting and inference steps in order to construct spatially resolved \textit{parametric maps}. To highlight potential benefits when considering individual characteristics, we close with a linear regression classification scheme predicting presence via individual characteristics.

%%%%% writing ressources
%As a concurrent effect, we present an easily adaptable and scalable GLM analyses framework to investigate movement behavior with a high spatial resolution.
%Here, we first confirm individual differences in movement profiles of experienced video gamers moving faster and more efficient when exploring a large-scale VR.


% 2 Zooming in (main point) / 3 Related Work
% Address the question: What were the previous solutions to the stated problem?
\section{2 Related Work}
To this day, human development is explainable by the need to interact with a physical reality. Growing up, the human brain \textit{presumably} develops as a model of the latent hidden variables governing the observable behavior of cause and effect in the environment \cite{Friston2010}. Understanding cognition as a predictive process means brains constantly compare what is effectively happening in the world, with what was predicted to happen \cite{Clark2013}, for example inferring the trajectory of a thrown ball and successfully catching it.

\subsection{Natural behavior as a precursor of presence experience}
Successful immersion into virtual worlds relies on matching expectations that were substantiated in the non-virtual physical reality. Assuming that to experience presence in virtual environments equals treating what you perceive as a part of the reality you are currently in, many researchers have argued for an increase in ecological validity through VR experimentation across the cognitive sciences \cite{Bohil2011, Parsons2015, Parsons2017}. The assumption hence is that participants under the influence of successful VR illusions experience presence and therefore behave \textit{realistically} or with higher ecological validity. This work exhibits an approach to quantify such claims with spatial specificity in order to guide future design decisions in research and application. We point out, that this approach is scalable to investigations of spatially resolved behavioral, psychometric and bio-physiological parameters such as controller jerk, hit accuracy for example in video games or electroencephalographic (EEG) parameters \cite{Gehrke2019}.

\subsection{Presence experience impacts behavior}
Generally speaking many studies have investigated participants in VR illusions employing (A) emotionally charged stimulus material or (B) embodiment illusions. Considering (A), \cite{Diemer2015} provide a thorough overview of the intricate interplay between presence and reactions to emotionally charged stimulation in VR. The authors observed a consistently reported link between presence and the emotional experience in VR. They argue, that by varying degrees of arousal, presence impacts psychological as well as physiological responses. Considering (B), we highlight one relevant aspect from the rich literature on the effects of full-body embodiment into avatars \cite{Maister2015}. The proteus effect characterizes behavioral perturbations depending on avatar body attributes. Yee et al. showed that participants being immersed into an 'attractive' avatar moved closer into the interpersonal space of another person \cite{Yee2007}. Further, Banakou et al. demonstrated that within participants, the size of objects was overestimated while embodied into a child body compared to a non-embodied baseline \cite{Banakou2013}. Furthermore, one of the key VR illusions contributing to the subjective construct of presence is the place illusion which can be defined as the perception of oneself being present in a virtual place where one can act, react, and impact the surroundings \cite{Slater2009}. Therefore, self-location, sense of agency, and spatial awareness of the surroundings are strongly impacted by the place illusion and modify behavior \cite{Kilteni2012}. When one perceives oneself in control of ones own actions and observe action consequences in the virtual surroundings, spatial exploration behavior becomes a part of a learning process to adapt motor behavior to the surroundings as opposed to a random chain of actions executed by the user to explore, for instance, only the VR technology itself \cite{Tan2011}. Together, these examples illustrate that presence experience directly impacts cognition, the action-perception cycle.

\subsection{Benefits of studying spatially resolved behavior}
% Summary motivating our work:
% - what is the benefit of spatial resolution?
% find examples of mocap analysis collapsing data to the mean
% - cite studies with interesting findings but not spatially resolved, i.e. grand-averaged
%Problems of prior research in terms of the resolution of behavioral parameters investigate with respect to proteus effect or other. collapsing behavioral parameters to singular events


%%%%% writing resources
%The framework is derived from the neuroscientific methodology of parametric brain imaging, see for example \cite{Friston1994, Pernet2011}.
%mental representation of the surroundings and respectively the motor behavior.
%Novel immersive paradigms in the behavioral cognitive sciences as well as neurosciences subject participants to controlled, yet stimulus rich experiences in head-mounted VR. 
%Therefore, to be able to employ VR and claim ecological validity the level of presence must be accounted for. 

% 4 User Study
\section{3 user study}
To promote our proposal of investigating moderators of VR experience via spatially resolved parametric maps, we challenge whether the level of experienced presence impacts spatial exploration behavior in a large VR. Our invisible maze task mimics a real-world exploration situation such as finding your way in complete darkness \cite{}. We conducted the following two-step analysis. First, we constructed parametric maps to assess where participants movement behavior was impacted as a function of experienced presence. To this end, we conducted mass-univariate pixel-by-pixel modeling of experienced presence on duration spent at each location, i.e. pixel. Zooming out, we followed up our findings with a classification scheme predicting experienced presence given per participant descriptors. We considered a data-driven approach to finding a useful, i.e. scalable, classification model. Therefore, we initially included a variety of participant descriptors subsequently reducing them to a set with highest explanatory power. In this way, we hope to present a minimal model so other researchers may easily reproduce our findings. 

%%%%% writing resources
%\cite{Gehrke2018}
%To sum up, we addressed how accurate subjectively reported presence may be predicted given a number of per participant descriptors.
%and the number of wall touches elicited there, the simple where and what of participants actions.
%Here, we propose a highly accurate subjective presence metric based on individual differences in video game experience, sex, and spatial perspective taking abilities. Furthermore, we show that the level of presence impacts free spatial exploration behavior in a large scale VR during a spatial navigation paradigm. Finally, using a methodological framework developed for the cognitive neuroscience, we showcase a powerful analysis framework to investigate ongoing behavior in cognitive paradigms in greater detail.
%Finally, using a methodological framework developed for the cognitive neuroscience, we showcase a powerful analysis framework to investigate ongoing behavior in cognitive paradigms in greater detail.

\subsection{Participants, Procedure, Task and Setup} Thirty-two healthy participants (aged 21--45 years, 14 men) took part in the experiment. All participants gave written informed consent to participation and the experimental protocol was approved by the local ethics committee (protocol: GR\_08\_20170428). Three participants were excluded from data analysis due to incomplete data or difficulties in complying with task requirements.

\begin{figure}[!t]
\centering
    \includegraphics[width=\linewidth]{figures/IMT_Task.png}
    %\vspace{0pt}
    \caption{Invisible Maze Task, \textbf{A} Participant from a bird’s eye view. \textbf{B} Participants are instructed to explore four different mazes and return to the start. \textbf{C} First-person view in binocular `VR optics' of a wall touch. \textbf{D} Top: Participants draw a top-down view of the explored maze. Participant is equipped with 160 channels wireless EEG, head-mounted virtual reality goggles and LEDs for motion capture. Bottom: drawn sketch map. Find a detailed description in \cite{Gehrke2018}}
    \label{imt_task}
\end{figure}

\subsubsection{The Invisible Maze Task} Participants freely explored a sparse invisible maze environment interactively by walking and probing for visual feedback when touching the virtual wall of a 1m wide path with their \textit{right} hand. All stimuli were presented using an Oculus Rift DK2 VR headset which was adaptable to be incorporated with the dedicated optical tracking system. Further the DK2 headset offered benefits regarding weight and modifiability. Upon collision of the \textit{right} hand with an invisible wall, a white disc was displayed 30 cm behind the collision point parallel to the invisible wall much like the beam form a torch in a cave, see figure \ref{imt_task} C (consult \cite{} for extensive details on the task, instrumentation, and data collection). We did not track the left hand and instructed participants to keep along their body. Performing the task, participants explored four different mazes in order, `I', `L', `Z' and `U', see f \ref{imt_task} B). The task was designed to emphasize participants internal build-up of a spatial representation of the maze layout. Doing the task, participants display a behavior that is comparable to explorative wall touches in the dark to find your way.

For each maze the procedure was as follows: participants were briefly disoriented and then positioned facing the open side of the path. Then, participants were directed to explore the invisible path until they reached a dead end, and subsequently find their way back to the starting position. At the end of each maze trial, participants received a gamified feedback and were then asked to draw a sketch map of the maze from a bird’s eye view as an index of spatial learning. The procedure was repeated three times in a row for each maze to foster spatial learning. For the present analysis, the three exploration phases were aggregated. The complete experiment, including preparation of physiological measures (Electroencephalogram, EEG) took approximately 4 hours. Preceding and following the task participants completed a set of questionnaires. We collected synchronized motion capture and behavioral events alongside high-density EEG.

\subsubsection{Assessing presence} To assess experienced presence, we administered the igroup presence questionnaire \cite{Schubert2003}. For the reported analyses we considered only the first item of the questionnaire, the subjective presence measure (G1) is representing the sense of being in a place, i.e. 'In the computer generated world I had a sense of "being there' rated from 'not at all' to 'very much' on a 7-point Likert scale \cite{Schubert2003, Slater1993}.

% anonymisation
%~\cite{Gehrke2018}
\subsection{Statistical Analyses}
Enabling our proposed analyses framework, two key challenges must be addressed. First, capturing (rigid body) motion in 3D. With state-of-the art VR hardware sampling motion data at around 90Hz accessing and recording pose data, position and orientation, is possible\footnote{https://brekel.com/openvr-recorder/}. Further libraries to synchronize data streams across the network exist\footnote{See for example https://github.com/sccn/labstreaminglayer and http://openvibe.inria.fr/, with predominant application across the neurosciences.} providing affordable alternatives to dedicated systems. Second, in order to compare pixel-wise motion data across participants, each participant should exhibit data points at each pixel. Therefore, decreasing spatial resolution by sub-sampling and/or smoothing can be employed to address the second challenge.

The proposed analyses approach can be summarized into three separate steps:
\begin{itemize}
    \item \textbf{Single-subject summary (or first-level).} With this analyses, we were interested in expressing where participants spent most of the time exploring the mazes as a function of experienced presence. Hence, we investigated the location in 2D (X,Y) of the VR headset rigid body, see \ref{imt_task}. To speed up subsequent analyses, we first sub-sampled the motion capture to 1Hz. Then we computed individual averages of these motion capture data (position over time) for each maze. We averaged across the three repeated explorations per maze, as we were not interested in changes over repeated trials but in expressing exploration behavior as a function of experienced presence more generally. With an average exploration duration of ~3 minutes, for each maze and participant ~180 samples were kept in x and y, discarding the third dimension (up-down) in this analyses, \ref{fig:methods} (left) shows one exploration phase of one subject with lines plotted to connect each sample.
    \item \textbf{Enabling group-level inference.} Next, a 2D histogram with fixed edges to maintain equal resolution across participants was computed. In order to increase overlap across participants (second challenge, see above) a 2D (square sized) Gaussian blur was applied to the histogram image. A sigma of 1.5 was chosen for the 2D filter kernel as it resulted in a good overlap across participants while maintaining spatial specificity.
    \item \textbf{Group-level inference (or second-level).} To investigate the impact of experienced presence on each parameter, we calculated a linear regression at each pixel of the map. We specified the model as $pixels \sim\ intercept + presence + error$, hence at each pixel we fit a linear regression across participants with presence scores as the predictor variable. Pixels with data of fewer than 12 participants (critically low N) were kept as `NaN' and not subjected to linear regression analyses. Plotting resulting regression estimates yields a 2D parametric map. Uncorrected p-values were used to plot a contour at significant effects with $p < .05$. The Matlab code used to construct the parametric maps is available online\footnote{https://github.com/lukasgehrke/mobi-3D-tools}. In order to keep the focus of this report on the potential opportunities and due to the exploratory nature of our investigation, we chose not to implement robust statistics and correction for multiple comparisons and direct interested readers elsewhere~\cite{Pernet2011, Wilcox2016a}. Furthermore, accurate correction for multiple comparison correction must consider the shape of the p-values map. In other words, some form of cluster-based statistic taking into account values of neighboring pixels is preferable. Such cluster-based statistics exist, for example cluster-mass and threshold-free cluster enhancement, but are not trivial to report~\cite{Pernet2015}.
\end{itemize}
% \subsubsection{Parametric mapping with movement behavior}
\subsubsection{Predicting Presence using Participant Descriptives} 
To follow up our spatially-resolved analyses and situate our findings more generally, we zoomed out and set out to predict presence scores using aggregate data and participant descriptors. We computed a least squares regression entering the IPQ presence (G1) scores as the dependent variable using R~\cite{RFoundationforStatisticalComputing.2018}. To accentuate the effect of aggregating motion capture data we, we included aggregates of movement velocity, time spent exploration or exploration duration. Additionally, we included the average number participants touched a wall with their hand as well as the average accuracy of a sketch map participants had to draw following each exploration. The last metric alludes to an accurate mental spatial representation, which was specifically queried with the \textit{invisible maze task}. We aggregated the measures by computing individual averages over mazes and runs. Further we added participants video game experience, sex, perspective taking and orientation ability and lastly the sense of direction into the regression model. For a detailed explanation of each predictor consult~\cite{Gehrke2018}. Predictors that do not significantly add to the explanatory power of the regression model were localized using a step-wise model selection procedure based on Akaike's information criterion (AIC). The procedure was computed using `stepAIC' of package `MASS'~\cite{Akaike1998a, Venables2002}. This data-driven procedure was selected to exclude the likely problem of over-fitting for the final reported model and to increase the usability of the approach by minimizing the number of included predictors. After the step-wise model selection, three predictors remained in the model predicting presence. Ultimately, the reduced model with three predictors was assessed in terms of its predictive accuracy. Therefore, a 5 fold cross-validation was computed to obtain a robust mean absolute error~\cite{Mosteller1968, Furnkranz2011}. With 29 participants, each training fold consisted of either 23 or 24 participants with either 5 or 6 participants in the evaluated test fold. The R code and data used are available online\footnote{anonymized}.
% for anonymization: https://github.com/lukasgehrke/2019-IEEE-VR-Spatial-exploration-behavior-in-large-scale-VR-predicts-subjective-spatial-presence

% 5 Results
\section{Results}
We confirmed that the level of experienced presence impacted spatial exploration behavior in VR. Interestingly the effect was similar across different mazes with a pattern of increasing presence associated with staying longer in the center of the maze and spending more time in segments \textit{presumably} critical for navigational success, i.e. corners. To emphasize the importance of considering individual differences when designing room-scale VR for a broad public, we carved out significant individual characteristics predicting the level of experienced presence, potentially of use for directing future design decisions.

\begin{figure}[h]
\centering
\includegraphics[width=\linewidth]{figures/head_loc_mean.pdf}
% \vspace{6pt}
\caption{Time spent at each location (grand-average) in each of the four mazes: `I', `L', `Z' and `U'. The whole lab space covered roughly 12 by 8 meters in size. Mazes are constructed of ten 1x1 meter squares. Warmer colors, i.e. red, indicate a longer time spent at that location. Green triangles indicate participants starting position.}
% \Description{Time spent at each location (grand-average) in each of the four mazes `I', `L', `Z' and `U' during exploration.}
\label{head_loc_mean}
\end{figure}
\subsection{Exploration behavior expressed as a function of presence} 
First, averaging spatially-resolved exploration time across participants revealed that participants spent more time in dead-ends as well as in areas including corners. The effect was observed across all mazes (see red colors in figure \ref{head_loc_mean}). Conversely, less time was spent when located in the straight segments of each maze, i.e. participants moved faster in those segments. As expected, participants also spent more time at the beginning of the trial, taking a moment to start moving. Together, these findings confirmed our intuitive expectations of the exploration behavior. 

Second, we investigated the effect of presence on time spent exploring mass-univariately for all locations in all mazes. With increasing presence scores, time spent at the center of the straight paths increased, for example in maze `I' at [3.5, 1.5] $beta=.027, t_{(28)}=2.76, p=.01, R^2=.23$ and `Z' at [1.8, 10] $beta=.039, t_{(28)}=3.34, p<.01, R^2=.34$, see figure \ref{presence_head_loc} A. Further, increasing presence scores correlated with more time spent at the center of the paths in corners, for example in maze `L' at [-1, 3.5] $beta=.037, t_{(28)}=2.33, p=.03, R^2=.19$ and `U' at [0.5, 17.5] $beta=.034, t_{(28)}=2.44, p=.02, R^2=.19$. Conversely, closer to the maze boundaries we observed a negative effect with increasing presence scores correlating with less time spent in these areas, specifically in the most challenging mazes `Z' at [2, 11.5] $beta=-.073, t_{(28)}=-4.16, p<.01, R^2=.40$ and `U' at [0.5, 17] $beta=-.143, t_{(28)}=-4.05 p<.01, R^2=.40$. Taken together, with an increase in experienced presence, participants spent more time firmly located at the center of the path, specifically along the straight segments in the `I' and `Z' maze as well in navigationally relevant corners of mazes `L', `Z' and `U'. Further underlining this observation, higher reported presence scores negatively correlated with the time spent close or even colliding with the maze walls.
\begin{figure}[!h]
\centering
\includegraphics[width=\linewidth]{figures/head_loc_reg2.pdf}
\vspace{6pt}
\caption{Map of the impact of presence on the time spent at every location (betas) in each of the four mazes: `I', `L', `Z', `U'. Green triangles indicate participants starting position. Warmer colors refer to a positive regression estimate. The values can be understood in the sense that for each 1 point increase in reported presence, participants stayed `z' seconds longer at location `xy'. Significant pixels at $p<0.05$ are masked with a dotted line}
\label{presence_head_loc}
\end{figure}

\subsection{Video game experience, biological sex and perspective taking predict presence experience} A step-wise model selection resulted in three remaining predictors, explaining 53,4\% of the variation in experienced presence ($F_{(3,25)}=11.69, p < .001,$ adjusted $R^2=.534$). Participants' predicted presence score was equal to $8.15 - 1.2 (Video Game Experience) + 2.61 (Biological Sex) - 0.02 (PTSOT)$ where biological sex was dummy-coded as 0 = Male, 1 = Female, increasing video game experience was coded with higher scores and decreasing perspective taking ability with higher scores. Video game experience ($t_{(25)}=-4.7, p<.001$), biological  sex ($t_{(25)}=5.6, p<.001$) as well as perspective taking ability ($t_{(25)}=-2.52, p=.02$) were significant predictors of presence. Cross-validation of the three predictor model above yielded a combined average 0.76 mean absolute error. Hence, using video game experience, biological sex as well as perspective taking ability we were able to predict experienced presence with a deviation of $\pm 0.75$ points from the true value on the `IPQ Likert scale'.

% 6 Discussion
\section{Contribution, Benefits \& Limitations}
With the presented approach we aim to promote benefits of greater spatial resolution of movement behavior to assess a variety of phenomena moderating, for example, the presence experience in VR. Here, a grand average of, e.g. time spent exploring mazes, provides only limited insights due to its spatial dependence. Some participants may spent more time in the corners but walk faster along straights segments while other participants might wander through the mazes with a constant walking speed, not adapting to the features of the environment. Without a two-dimensional resolution, the two different behaviors would average out to the same amount of time spent in a maze. Therefore, we constructed spatial parametric maps mimicking established data analyses procedures in cognitive neuroscience. For the ground work motivating our proposal, as well as a discussion of the state-of-the-art methods linking behavior to brain activity across the cognitive neurosciences consult~\cite{Friston1994b, Bridwell2018a}. With this paper, we hope to motivate developers and researchers to use parametric mapping using the inherent motion capture capabilities of VR technology for a data-driven understanding of user behavior and, ultimately, user experience. We introduced parametric maps, expressing VR exploration behavior as a function of experienced presence as an exemplary predictor variable related to the subjective user experience. 

In the user study we showed that increasing subjective presence scores coincided with an increased time spent at the center of the paths' through our simple \textit{invisible} mazes. In turn, this effect translated to participants with high presence scores spending less time closer to the walls, or crashing into them, presumably relating to high spatial awareness. The spatial specificity of our reported effect accentuates that aggregate behavioral metrics, such as average speed or time spent across one maze exploration, may miss critical information. Here a finer resolution has proven to be beneficial.

Regardless, our analyses exposed that high presence scores coincided with more time spent in the corners, further underlining the hypotheses of increasing spatial awareness with increasing presence. Sense of embodiment has frequently been included as a key component in psychological models of presence experience with a sense of self-location as one emergence of embodiment~\cite{Kilteni2012}. This sense of a virtual self is a logical prerequisite for spatial anchoring and reduced cognitive effort when imagining a third-person allocentric perspective of space. Here additional mental transformations potentially disrupt the presence experience by violating the \textit{natural} experience of the predictive brain~\cite{Gonzalez-Franco2017}. In consequence, solving spatial task, for example requiring perspective taking, depend on individual abilities and preferences for specific spatial strategies but appear to be `easier' with a congruent sense of embodiment~\cite{Pan2018, Gramann2013, Gramann2005, Cognition2016, Jeung2022-lt, Gramann2021-ug, Delaux2021-ph}.

Using participants individual characteristics, we were able to predict presence scores with an average deviation of $\pm 0.75$ from the true reported score. Employing a data-driven model selection procedure determined that apart from video game experience, biological sex as well as perspective taking ability, no other predictor significantly contributed to the linear regression scheme. Interestingly, increasing video game experience negatively impacted presence scores, conflicting with previous findings~\cite{Lachlan}. We hypothesize that this could be explained by the overly sparse visuals of the virtual environment in the \textit{invisible maze task}. Participants with significant video gaming experience might perceive the world as too reduced and artificial compared to usually rich rendering of video games. With regards to the impact of biological sex, we contribute one more example of contradictory results in the literature~\cite{Coluccia2004}. Based on our outcome, however, we cannot argue that the underlying differences in the task were due to a spatial component or rather the interaction with an unknown sparse virtual environment and refer to a detailed discussion on the topic~\cite{Felnhofer}. Ultimately, perspective taking ability had a limited influence on predicting subjective presence. Increasing perspective taking scores, referring to an angular error in a mental triangulation task, negatively impacted the subjective feeling of presence. We hypothesize that the experience of presence may arise earlier in individuals that feel oriented and that are aware of their spatial surroundings. As such, spatial awareness is useful when solving tasks in unfamiliar environments~\cite{Slater2018}.

\section{Conclusion}
With this work we motivate continuous, spatially resolved, analyses to understand individual characteristics explaining exploration behavior in VR, ultimately explaining user experience. However, user experience is subjective and traditionally assessed via questionnaires following the experience to be evaluated. The here presented approach exhibits several advantages. First, a continuous assessment is desirable over a discrete sampling after the VR experience and second, using active and continuous behavioral parameters without interrupting the ongoing experience to administer the questionnaire allows for a non-distorted assessment of presence.

\subsection{Towards an unobtrusive, spatially resolved understanding of user experience}
With this work we showed that observing finely resolved overt behavior may provide insights into the subjective user experience, such as presence experience. However, using post-experience questionnaires to `measure' presence experience is problematic~\cite{Slater2004-rn}. Recently, increasing efforts have been made to investigate the physiological basis of presence experience in real-time. Using physiological methods holds the potential to overcome indirect, post-experience, measurements, allowing to directly assess the physiological source that realizes the subjective experience, i.e. the brain~\cite{Gehrke2019, Singh2018, Si-mohammed2020, Gehrke2022-tj}. Gehrke et al. (2019, 2022) demonstrated that neural responses associated with mental error-processing following visuo-haptic mismatches in VR can be detected using EEG~\cite{Gehrke2019, Gehrke2022-tj}. Si-Mohammed and colleagues (2020) were able to use this EEG signature in a brain-computer interface classification scheme exhibiting high classification accuracy in a similar paradigm exhibiting graphical glitches in VR. These promising findings, hint at future applications in room-scale scenarios, picking up scenarios where a loss of immersion occurs in near real-time. Here, additional wearable sensors can supplement brain recordings to, for example, measure affective responses~\cite{Marin-Morales2018}.

To sum up, we propose a method to investigate user experience in VR by using a spatially resolved analyses approach of ongoing behavior, providing designers and researchers a tool to guide future design decisions. Investigating how individual characteristics and the current spatial context influences user experience provides the opportunity for inclusive design decisions driving VR technology acceptance across the broad public. Ultimately, we believe our approach can further be nurtured by unobtrusive, continuous (neuro-)physiological measurements for a multi-dimensional assessment of user experience.


%%%%% writing ressources
% of course modeling can be done the other way around, for example predicting the subjective workload experience based on where people look, and the results presented here can be read in this way. However we deliberately framed our analyses to emphasize on behavior recovered through motion capture.

%\section{Acknowledgments}
%We thank Jonna Jürs, Nora Moser for assisting with the data collection, Timotheus Berg for reviewing the methods section and Sezen Akman for fruitful discussions throughout.
%Authors 1 \& 2 gratefully acknowledge the grant \#01GQ1511 awarded to Author 2 from the German Federal Ministry of Education and Research (BMBF; Bundesministerium für Bildung und Forschung).

% \section{Acknowledgments}

% Identification of funding sources and other support, and thanks to
% individuals and groups that assisted in the research and the
% preparation of the work should be included in an acknowledgment
% section, which is placed just before the reference section in your
% document.

% This section has a special environment:
% \begin{verbatim}
%   \begin{acks}
%   ...
%   \end{acks}
% \end{verbatim}
% so that the information contained therein can be more easily collected
% during the article metadata extraction phase, and to ensure
% consistency in the spelling of the section heading.

% Authors should not prepare this section as a numbered or unnumbered {\verb|\section|}; please use the ``{\verb|acks|}'' environment.

% \section{SIGCHI Extended Abstracts}

% The ``\verb|sigchi-a|'' template style (available only in \LaTeX\ and
% not in Word) produces a landscape-orientation formatted article, with
% a wide left margin. Three environments are available for use with the
% ``\verb|sigchi-a|'' template style, and produce formatted output in
% the margin:
% \begin{itemize}
% \item {\verb|sidebar|}:  Place formatted text in the margin.
% \item {\verb|marginfigure|}: Place a figure in the margin.
% \item {\verb|margintable|}: Place a table in the margin.
% \end{itemize}

%%
%% The acknowledgments section is defined using the "acks" environment
%% (and NOT an unnumbered section). This ensures the proper
%% identification of the section in the article metadata, and the
%% consistent spelling of the heading.
% \begin{acks}
% To Robert, for the bagels and explaining CMYK and color spaces.
% \end{acks}

%%
%% The next two lines define the bibliography style to be used, and
%% the bibliography file.
\bibliographystyle{ACM-Reference-Format}
\bibliography{bibliography}

%%
%% If your work has an appendix, this is the place to put it.
% \appendix

% \section{Research Methods}

% \subsection{Part One}

% Lorem ipsum dolor sit amet, consectetur adipiscing elit. Morbi
% malesuada, quam in pulvinar varius, metus nunc fermentum urna, id
% sollicitudin purus odio sit amet enim. Aliquam ullamcorper eu ipsum
% vel mollis. Curabitur quis dictum nisl. Phasellus vel semper risus, et
% lacinia dolor. Integer ultricies commodo sem nec semper.

% \subsection{Part Two}

% Etiam commodo feugiat nisl pulvinar pellentesque. Etiam auctor sodales
% ligula, non varius nibh pulvinar semper. Suspendisse nec lectus non
% ipsum convallis congue hendrerit vitae sapien. Donec at laoreet
% eros. Vivamus non purus placerat, scelerisque diam eu, cursus
% ante. Etiam aliquam tortor auctor efficitur mattis.

% \section{Online Resources}

% Nam id fermentum dui. Suspendisse sagittis tortor a nulla mollis, in
% pulvinar ex pretium. Sed interdum orci quis metus euismod, et sagittis
% enim maximus. Vestibulum gravida massa ut felis suscipit
% congue. Quisque mattis elit a risus ultrices commodo venenatis eget
% dui. Etiam sagittis eleifend elementum.

% Nam interdum magna at lectus dignissim, ac dignissim lorem
% rhoncus. Maecenas eu arcu ac neque placerat aliquam. Nunc pulvinar
% massa et mattis lacinia.

\end{document}
\endinput
%%
%% End of file `sample-authordraft.tex'.
